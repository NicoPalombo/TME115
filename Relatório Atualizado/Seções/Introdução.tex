\subsection{\textbf{O que são criptomoedas?}}

Por definição, criptomoedas são qualquer tipo de moeda digital ou virtual que utiliza criptografia para garantir a realização de transações. Elas não têm autoridade central de emissão e sem regulação. Em contrapartida, elas utilizam um sistema de criptografia chamado de blockchain, além de outros sistemas de criptografia descentralizados para registrar transações e emitir novas unidades. 
\textbf{Blockchain é um banco de dados distribuído que consegue fazer um} compartilhamento de informações dentro de uma rede. Essas informações podem ser variadas, agrupadas em bloco a partir de criptografia, de uma forma que consiga trazer consigo o histórico completo e imutável das transações anteriores. Garantindo uma rede de operações concisas, seguras e confiáveis para qualquer um que desejar entrar nessa rede.

As criptomoedas atuam com diversas funções e objetivos além de um simples meio de transação monetária para uso diário. Elas podem ser reserva de valores; atuação como \textbf{ações de investimento}, por conta de sua alta volatilidade de seu valor; descontos em taxas; garantia de algum serviço; entre muitas outras funções.

Um dos maiores exemplos de criptomoedas são o Bitcoin e o Ether. O \textit{Ether} valendo cerca de \textbf{2.55 mil dólares americanos}, além do \textit{Bitcoin}, tendo mais de \textbf{100 mil dólares americanos} e sendo a criptomoeda mais famosa do mundo.

\subsection{\textbf{O que é \textit{Bitcoin}?}}
Lançado em 2008, com o pseudônimo de Satoshi Nakamoto, uma nova ideia foi criada e posta ao mundo, com a criação do Bitcoin. O Bitcoin é atualmente a criptomoeda mais famosa do mundo, além de ser a mais valorizada atualmente, ultrapassando a marca de US\$100.000,00 

Devido ao fato do Bitcoin ser uma criptomoeda, logo, ela não tem uma emissão regularizada e centralizada, a emissão de novas moedas no mercado é de uma outra forma. É usado o termo "mineração". Esse processo consiste na resolução de quebra-cabeças matemáticos complexos feitos por hardware e software especializados para a mineração de Bitcoin. A mineração não é exclusiva de um grupo de pessoas, empresas ou bancos. Contanto que tenha energia e um hardware de mineração, é possível obter um bloco de Bitcoin.

Os blocos de Bitcoin são os produtos e agrupamentos de transações individuais dos últimos dez minutos de cada mineração. Cada bloco é único e fechado, cada um criando seu próprio número \textit{Hash}, que é onde estão codificadas as transações. Cada novo bloco precisa ter em seu conteúdo as informações do bloco anterior, o que garante uma veracidade de que não será manipulado ou alterado. Cada bloco gerado é equivalente a uma quantidade específica de Bitcoin, cada bloco tem atualmente 3.125 BTC (Bitcoin).

Com tal unicidade, garante uma individualidade perante cada unidade de moeda não seja possuída por mais de um portador. O que torna a concorrência de cada máquina para garantir seu bloco próprio maior ainda, incentivando cada vez mais a melhora dos computadores para fazer a mineração. Além disso, para a autenticação no sistema de Prova de Trabalho (PoW) do Bitcoin, os computadores de mineração precisam comprovar a energia gasta para a mineração de cada bloco. O que reforça a segurança e autenticidade de cada máquina e bloco minerado.

% Gráfico Principal
\begin{figure}
    \centering
    \includegraphics[width=1\linewidth]{Imagens/Grafico-Princial.jpg}
    \caption{Histórico do Valor do Bitcoin}
    \label{fig:Candle Graph}
\end{figure}

\subsection{\textbf{Eventos Importantes:}}

A lei da oferta e demanda tem grande influencia sobre a valorização e  desvalorização do Bitcoin.  Assim como nos mercados tradicionais, quando a oferta de um produto é limitada e sua demanda aumenta, o seu valor tende a aumentar também. Só que no casso do Bitcoin, ele é tanto o produto quanto uma possível moeda de troca externa. Isso tudo é realçado por ter uma oferta limitada à no máximo 21 milhões de unidades. No entanto, também existe outros fatores que pode influenciar  em seu valor, tendo os principais exemplos: o evento Halving, o caso da Mt. Gox e a pandemia  de COVID-19.

Ocorrendo aproximadamente a cada 4 anos, o evento Halving  reduz pela metade o ganho de Bitcoin por bloco minerado, diminuindo a entrada de novos bitcoins no mercado. Este evento é associado geralmente com a valorização do bitcoin, já que aumenta a escassez da moeda, favorecendo a sua valorização.

Acontecimentos externos também podem ocasionar a desvalorização da moeda, como por exemplo a falência da empresa Mt. Gox, em 2014. Na época, a maior corretora de criptomoedas do mundo, acabou perdendo por volta de 800 mil bitcoins após sofrer uma sequência de invasões cibernéticas. Causando uma queda brusca no valor do bitcoin e afetando a confiança neste mercado.

A pandemia de COVID-19 em 2020 também teve um grande impacto no valor do bitcoin. Com a incerteza global ao início da pandemia, o Bitcoin apresentou uma queda em seu valor, porém com a desvalorização das moedas fiduciárias, a bitcoin passou a ser procurada como investimento alternativo, resultando em uma significativa valorização.