Esse trabalho aborda uma análise sobre o Bitcoin e seu valor  como uma nova forma de investimento. Foram destacados e exemplificados eventos importantes que impactaram a alteração na valorização no decorrer da história do Bitcoin, como a falência da Mt. Gox, a reação durante a pandemia de COVID-19 e sua influência no decorrer de cada evento Halving. Utilizando o banco de dados “Bitcoin Historical Data” obtido da plataforma \texttt{Kaggle}, separando suas informações em intervalos de um minuto. Foram feitas análises estatísticas e gráficas para visualizar e evidenciar sobre o comportamento humano lidando com imprevistos indesejados ou acontecimentos previsíveis. Análises mostram que a relação de correlação entre volume de transações e retorno diário, mostrando que acontece um efeito “Manada”, sendo positiva em dias de altas e negativa em dias de baixa. Esses comportamentos refletem que, quanto mais é vendido cada unidade, tende a cada vez mais pessoas a venderem mais, e o contrário também se aplica a isso. Além disso, é bem aparente e reforçado que o Bitcoin atua como um ativo financeiro alternativo, podendo ser um resistente artifício para lidar com as mais diversas crises globais.  