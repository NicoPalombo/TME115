A analise dos dados histórico do Bitcoin revelam parte da psique do ser humano em relação a como se lida com mercadorias. 
Os testes de correlação mostram visualmente como funciona: enquanto em dias de alta há uma correlação positiva entre volume e retorno, nos dias de baixa essa correlação é negativa, indicando pressões de venda mais intensas. O que mostra que há um efeito "Manada", em que por justamente está crescendo o seu valor, mais pessoas querem comprar para vender mais caro. E o oposto acontece também, quando há uma tendência  de vendas, todos já correm para vender ainda com o preço alto para não perderem dinheiro. 

E como mostrado nos eventos analisados, com um acontecimento (previsível ou não) amplifica esse efeito. Com a alta variação frequente do valor do Bitcoin, eventos extremos como a falência da Mt. Gox e a pandemia mostram uma sensibilidade sobre imprevistos e como o ser humano lida com ele. Em contra partida, vemos também para quando um evento previsível vai acontecer mostrado com o evento Halving.
Em conjunto com tudo, esses padrões mostram uma parte do comportamento humano, além de reforçar a sensibilidade sobre valor do Bitcoin.
Por fim, destaca-se o papel crescente do Bitcoin como um ativo alternativo em cenários econômicos instáveis.
